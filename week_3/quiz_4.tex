% Created 2024-06-23 Sun 01:56
% Intended LaTeX compiler: pdflatex
\documentclass[11pt]{article}
\usepackage[utf8]{inputenc}
\usepackage[T1]{fontenc}
\usepackage{graphicx}
\usepackage{longtable}
\usepackage{wrapfig}
\usepackage{rotating}
\usepackage[normalem]{ulem}
\usepackage{amsmath}
\usepackage{amssymb}
\usepackage{capt-of}
\usepackage{hyperref}
\usepackage[a4paper,left=1cm,right=1cm,top=1cm,bottom=1cm]{geometry}
\usepackage[american]{babel}
\usepackage{enumitem}
\usepackage{float}
\usepackage[sc]{mathpazo}
\linespread{1.05}
\renewcommand{\labelitemi}{$\rhd$}
\setlength\parindent{0pt}
\setlist[itemize]{leftmargin=*}
\setlist{nosep}
\newcommand{\dlog}[3]{\mathrm{dlog}_{#2,#3}\:#1}
\author{Marcio Woitek}
\date{}
\title{Discrete Logarithm and Primitive Root}
\hypersetup{
 pdfauthor={Marcio Woitek},
 pdftitle={Discrete Logarithm and Primitive Root},
 pdfkeywords={},
 pdfsubject={},
 pdfcreator={Emacs 29.3 (Org mode 9.6.24)}, 
 pdflang={English}}
\begin{document}

\maketitle
\thispagestyle{empty}
\pagestyle{empty}

\section*{Problem 1}
\label{sec:org45c2690}
\begin{equation}
\dlog{3}{2}{5}=3
\end{equation}

\section*{Problem 2}
\label{sec:org2f08df2}
\begin{equation}
\dlog{4}{5}{7}=2
\end{equation}

\section*{Problem 3}
\label{sec:org19581df}
\begin{itemize}
\item 2
\end{itemize}

\section*{Problem 4}
\label{sec:org6d62e76}
\begin{itemize}
\item 3
\item 5
\end{itemize}

\section*{Problem 5}
\label{sec:org28488b7}
\begin{itemize}
\item Given a large modulus \(n\), the discrete logarithm problem is computationally
difficult.
\item Using the primitive roots of a prime modulus \(p\) yields the maximum \(p-1\)
possible outcome values for the discrete logarithm, which is desired for
cryptography.
\end{itemize}
\end{document}
