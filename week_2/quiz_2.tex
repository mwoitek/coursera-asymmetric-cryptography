% Created 2024-06-22 Sat 23:03
% Intended LaTeX compiler: pdflatex
\documentclass[11pt]{article}
\usepackage[utf8]{inputenc}
\usepackage[T1]{fontenc}
\usepackage{graphicx}
\usepackage{longtable}
\usepackage{wrapfig}
\usepackage{rotating}
\usepackage[normalem]{ulem}
\usepackage{amsmath}
\usepackage{amssymb}
\usepackage{capt-of}
\usepackage{hyperref}
\usepackage[a4paper,left=1cm,right=1cm,top=1cm,bottom=1cm]{geometry}
\usepackage[american]{babel}
\usepackage{enumitem}
\usepackage{float}
\usepackage[sc]{mathpazo}
\linespread{1.05}
\renewcommand{\labelitemi}{$\rhd$}
\setlength\parindent{0pt}
\setlist[itemize]{leftmargin=*}
\setlist{nosep}
\newcommand{\Mod}{\:\mathrm{mod}\:}
\author{Marcio Woitek}
\date{}
\title{RSA Operations}
\hypersetup{
 pdfauthor={Marcio Woitek},
 pdftitle={RSA Operations},
 pdfkeywords={},
 pdfsubject={},
 pdfcreator={Emacs 29.3 (Org mode 9.6.24)}, 
 pdflang={English}}
\begin{document}

\maketitle
\thispagestyle{empty}
\pagestyle{empty}

\section*{Problem 1}
\label{sec:org734283a}
\textbf{Answer: 4}\\[0pt]

When the consider the integers \(k\) in the range \(1\leq k\leq 12\), there are only 4
integers that are relatively prime to 12: 1, 5, 7, 11. Hence:
\begin{equation}
\varphi(12)=4.
\end{equation}

\section*{Problem 2}
\label{sec:org5b6634d}
\textbf{Answer: 40}

\begin{align}
\varphi(n)&=\varphi(pq)\\
&=(p-1)(q-1)\\
&=(5-1)(11-1)\\
&=40
\end{align}

\section*{Problem 3}
\label{sec:org54a212f}
\textbf{Answer: 14}\\[0pt]

Assuming the message is encrypted using the public key, the ciphertext is given by
\begin{equation}
C=M^e\Mod n.
\end{equation}
By substituting the known values, we get
\begin{equation}
C=9^3\Mod 55.
\end{equation}
First, we use that \(9^3=729\). Next, we write this power as \(729=13\cdot 55+14\).
Hence:
\begin{equation}
C=14.
\end{equation}

\section*{Problem 4}
\label{sec:org4fe052e}
\textbf{Answer: 60}

\begin{align}
\varphi(n)&=\varphi(pq)\\
&=(p-1)(q-1)\\
&=(7-1)(11-1)\\
&=60
\end{align}

\section*{Problem 5}
\label{sec:org4877588}
\textbf{Answer: 57}\\[0pt]

Assuming the message is encrypted using the public key, the ciphertext is given by
\begin{equation}
C=M^e\Mod n.
\end{equation}
By substituting the known values, we get
\begin{equation}
C=8^{17}\Mod 77.
\end{equation}
First, we use that \(8^{17}=2251799813685248\). Next, we write this power as
\begin{equation*}
2251799813685248=29244153424483\cdot 77+57.
\end{equation*}
Hence:
\begin{equation}
C=57.
\end{equation}
\end{document}
