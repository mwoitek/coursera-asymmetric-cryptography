% Created 2024-06-23 Sun 00:45
% Intended LaTeX compiler: pdflatex
\documentclass[11pt]{article}
\usepackage[utf8]{inputenc}
\usepackage[T1]{fontenc}
\usepackage{graphicx}
\usepackage{longtable}
\usepackage{wrapfig}
\usepackage{rotating}
\usepackage[normalem]{ulem}
\usepackage{amsmath}
\usepackage{amssymb}
\usepackage{capt-of}
\usepackage{hyperref}
\usepackage[a4paper,left=1cm,right=1cm,top=1cm,bottom=1cm]{geometry}
\usepackage[american]{babel}
\usepackage{enumitem}
\usepackage{float}
\usepackage[sc]{mathpazo}
\linespread{1.05}
\renewcommand{\labelitemi}{$\rhd$}
\setlength\parindent{0pt}
\setlist[itemize]{leftmargin=*}
\setlist{nosep}
\newcommand{\Mod}{\:\mathrm{mod}\:}
\author{Marcio Woitek}
\date{}
\title{RSA Algorithm}
\hypersetup{
 pdfauthor={Marcio Woitek},
 pdftitle={RSA Algorithm},
 pdfkeywords={},
 pdfsubject={},
 pdfcreator={Emacs 29.3 (Org mode 9.6.24)}, 
 pdflang={English}}
\begin{document}

\maketitle
\thispagestyle{empty}
\pagestyle{empty}

\section*{Problem 1}
\label{sec:org700870a}
\begin{itemize}
\item \(d\)
\item \(p\)
\item \(q\)
\item The Euler totient function of \(n\), \(\varphi(n)\)
\end{itemize}

\section*{Problem 2}
\label{sec:org2dd71e5}
\begin{itemize}
\item After choosing \(d\), the extended Euclidean algorithm can be used to derive \(e\).
\item After choosing \(e\), the extended Euclidean algorithm can be used to derive \(d\).
\item For the public-private keys of RSA, \(e\) and \(d\), given any plaintext
\(m\), \(m\) raised to the power of \(e\cdot d\) \(\left(m^{e\cdot d}\right)\)
is equal to \(m\).
\end{itemize}

\section*{Problem 3}
\label{sec:orgf66df00}
\begin{itemize}
\item 9
\item 17
\item 21
\end{itemize}

\section*{Problem 4}
\label{sec:org3645bed}
\textbf{Answer: 5}\\[0pt]

We can determine the original plaintext \(m\) with the aid of the following equation:
\begin{equation}
m=\frac{m^{\prime}}{r}.
\end{equation}
We were given the value of \(m^{\prime}\): \(m^{\prime}=15\). So we need to find \(r\).
To do so, we use the fact that the chosen ciphertext can be written as
\begin{equation}
c^{\prime}=cr^e\Mod n=14r^7\Mod 33=14\cdot 2187\Mod 33,
\end{equation}
where we've used the other information given in the problem statement. For the
last equality to hold, we must have
\begin{equation}
r^7=2187\Rightarrow r=3.
\end{equation}
Hence:
\begin{equation}
m=\frac{15}{3}=5.
\end{equation}
\end{document}
